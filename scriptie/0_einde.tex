%
% Conclusie
%

\chapter{Conclusie}

% Introductie
Het doel van deze masterproef was de ontwikkeling van een modern multimediaframework dat zorgt voor de opslag, distributie en weergave van multimediafragmenten op de kiosken van de volkssterrenwacht MIRA. Daarbij was belangrijk dat een beheerder elk van die processen voldoende kan opvolgen en beïnvloeden, zonder al te veel technische kennis nodig te hebben. Ook moest het mogelijk zijn om mooie voorstellingen te realiseren met de middelen die het platform aanbiedt, en tegelijk compatibel te zijn met de voorstellingen die momenteel gebruikt worden.

% Voorstellingen
We hebben ervoor gekozen om de voorstellingen op te maken in \ac{html} en Javascript, waarmee we zeer rijke gebruikerservaringen kunnen bekomen en de gegevens daarbij efficiënt kunnen opslaan en versturen over het netwerk. De keuze voor zulke recente technologieën was wel niet altijd een voordeel: omdat er nog geen \emph{content creation applications} voor bestaan is het voor niet-programmeurs moeilijk om voorstellingen te maken. Langs de andere kant is compatibiliteit met de huidige voorstellingen er zeer eenvoudig mee: na conversie naar het WebM formaat kunnen we de fragmenten direct gebruiken binnen een \ac{html} pagina.

% Repository
Vervolgens worden de voorstellingen opgeslagen in een \ac{svn} repository, waardoor efficiënte overdracht bekomen wordt en een beheerder relatief eenvoudig aanpassingen kan maken. Bij gebrek aan goed geïntegreerde Java bindings voor de \ac{svn} bibliotheek hebben we moeten gebruik maken van een externe \ac{svn} server, waardoor de complexiteit van het systeem nodeloos toeneemt. We hopen daarom dat we in een latere versie de repository volledig zullen kunnen integreren, waardoor de applicatie een alleenstaand geheel zal vormen.

% Configuratie
Om de instellingen van elke kiosk niet op het toestel zelf te moeten opslaan, hebben we er voor gekozen om ze ook in de repository op te slaan. De bestanden zijn daarbij opgemaakt als een \ac{xml} document, een krachtig formaat dat we eenvoudig kunnen valideren en verwerken. Momenteel kunnen instellingen enkel persistent gewijzigd worden door die bestanden handmatig te bewerken, maar hiervoor kunnen we later eventueel een applicatie ontwerpen om dit proces gebruiksvriendelijker te maken.

% Server
De serverapplicatie hebben we geschreven in Java, waardoor we toegang hadden tot verschillende uitgebreide bibliotheken om de implementatie van verschillende objectieven sterk te vereenvoudigen. Hoewel we daarbij verschillende problemen hebben gehad, zoals met de slecht functionerende JmDNS bibliotheek, en uiteindelijk code hebben moeten herschrijven wegens de incompatibiliteit met Java 1.5, kunnen we uiteindelijk spreken van een goed functionerende serverapplicatie die zijn taken op een robuuste manier uitvoert.

% Serverhardware
Als serverhardware hebben we gekozen om gebruik te maken van een Synology DS-207+, een \ac{nas} met de focus op laag energieverbruik en veilige dataopslag. Toch is het geen ideaal toestel, vooral omdat we gebruik moesten maken van het originele besturingssysteem dat op verschillende vlakken problemen opleverde met de serverapplicatie. Daarom zullen we wellicht op termijn overschakelen naar meer generieke serverhardware, waarop we een besturingssysteem naar keuze kunnen installeren.

% Client
Om een compacte maar toch cross-platform clientapplicatie te ontwikkelen, hebben we gebruik gemaakt van het Qt framework. Hierdoor hebben we ook meerde bestaande bibliotheken kunnen hergebruiken, al hebben we aan verschillende bibliotheken daarvan wijzigingen moeten doorvoeren om een goed werkende applicatie te bekomen. Om een compactere codebase te bekomen hebben we er ook voor gekozen om de hele applicatie-interface op te bouwen in \ac{html}, aangezien we voor de weergave van de voorstellingen toch al moesten voorzien in een \ac{html} engine.

% Clienthardware
Het toestel om de clientapplicatie op uit te voeren is na lang zoeken de Genesi Efika MX geworden, een relatief goedkope maar toch krachtige \emph{embedded computer}. Om er eenvoudig software op te kunnen installeren en onderhouden hebben we een op Debian gebaseerd besturingssysteem samengesteld waarbij onze eigen pakketten ingeladen worden van een lokaal beheerde repository. Op termijn hopen we het besturingssysteem nog te kunnen optimaliseren door gebruik te maken van de \code{armhf} variant.

% Inputmodule
Om gebruikersinput op de knoppen van de kiosk te kunnen verwerken hebben we ook een inputmodule gemaakt die de signalen doorstuurt via de \ac{usb} poort. Hiervoor hebben we een printplaat ontworpen, met een AVR microcontroller die instaat voor de verwerking van de signalen. We hebben de prijs ervan laag gehouden door gebruik te maken van de V-USB bibliotheek, dewelke een puur softwarematige \ac{usb} implementatie voorziet. Om geen extra drivers nodig te hebben op de client laten we de chip zichzelf registreren als een toetsenbord, en worden de signalen doorgestuurd als toetsaanslagen.

% Slot
We kunnen stellen dat het resultaat van deze masterproef een functioneel geheel vormt dat voldoet aan de opgelegde eisen. Logischerwijs kan er nog veel verbeterd worden, maar gaat dat grotendeels over niet-kritieke elementen die de essentiële werking van het systeem niet compromitteren. Het project zal dan ook in zijn huidige staat in het museum geïnstalleerd worden, al zal dat initieel slechts gaan om een enkel prototype. Gebaseerd op de feedback daarop zullen we wellicht na enkele maanden de nodige wijzigingen aan het systeem doorvoeren, om vervolgens over te gaan op de finale installatie op alle kiosken in het museum.

\clearpage


%
% Lijst met afbeeldingen
%

\listoffigures

\clearpage


%
% Lijst met fragmenten
%

\renewcommand{\lstlistlistingname}{Lijst van fragmenten}
\renewcommand{\lstlistingname}{Fragment}
\markboth{List van fragmenten}{Lijst van fragmenten}
\addcontentsline{toc}{chapter}{Lijst van fragmenten}
\lstlistoflistings
% TODO: probleem als meer dan 1 pagina aan fragmenten

\clearpage


%
% Bibliografie
%

\bibliographystyle{plainnat}
\bibliography{verslag}


%
% Blanco pagina
%

\newpage
\thispagestyle{empty}
\mbox{}
