%
% Conclusie
%


\chapter{Conclusie}

% todo: mag niet teveel op abstract lijken
% http://www.dissertationswriting.co.uk/dissertation_conclusion.htm 

Het doel van deze thesis was de ontwikkeling van een modern multimediaframework dat zorgt voor de opslag, distributie en weergave van multimediafragmenten op de kiosken van de volkssterrenwacht MIRA. Daarbij is belangrijk dat een beheerder elk van die processen voldoende kan opvolgen en beïnvloeden, zonder al teveel technische kennis nodig te hebben. Ook moet het mogelijk zijn om mooie voorstellingen te realiseren met de middelen die het platform aanbiedt, en tegelijk compatibel te zijn met de voorstellingen die momenteel gebruikt worden.

We hebben ervoor gekozen om de voorstellingen op te maken in \ac{html} en Javascript, waarmeer we zeer rijke gebruikerservaringen kunnen ontwerpen. Ook compatibiliteit met de huidige voorstellingen is ermee mogelijk: na conversie naar het WebM formaat kunnen we de fragmenten gebruiken binnen \ac{html}. Om daarbij dynamisch de taalkeuze van de gebruiker in te verwerken hebben we voorzien in een kleine webapplicatie, opgebouwd met behulp van JQuery.
Vervolgens worden de voorstellingen opgeslaan in een \ac{svn} repository, waardoor efficiënte overdracht bekomen wordt en een beheerder relatief eenvoudig wijzigingen kan doorvoeren aan de inhoud van de voorstellingen.

% server

% client

% inputmodule

% conclusie




%
% Lijst met afbeeldingen
%

\listoffigures

\clearpage


%
% Lijst met fragmenten
%

\renewcommand{\lstlistlistingname}{Lijst van fragmenten}
\renewcommand{\lstlistingname}{Fragment}
\markboth{List van fragmenten}{Lijst van fragmenten}
\addcontentsline{toc}{chapter}{Lijst van fragmenten}
\lstlistoflistings
% TODO: probleem als meer dan 1 pagina aan fragmenten

\clearpage


%
% Bibliografie
%

\bibliographystyle{plainnat}
\bibliography{verslag}
