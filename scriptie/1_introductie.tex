\part{Introductie}
\label{introductie}


%
% Doelpubliek
%

\chapter{Doelpubliek}
\label{introductie:doelpubliek}

\textit{Dit hoofdstuk zal beschrijven wat er interessant is aan deze scriptie, en voor wie dat nuttig kan zijn.}


%
% Motivatie
%

\chapter{Situatie}
\label{introductie:situatie}

De volkssterrenwacht MIRA biedt aan zijn bezoekers een uitgebreide rondleiding over sterrenkunde en aanverwanten. Aangezien het overgrote deel van de bezoekers bestaat uit kinderen (de sterrenwacht wordt vaak bezocht in schoolverband, of door families), doet men sinds jaar en dag moeite om de rondleiding zo interessant mogelijk te maken, ook wanneer er geen gids is om dat te verzorgen. Daartoe heeft men enkele jaren geleden een project gelanceerd, genaamd \emph{Ad-Astra}. Het doel van dit project was om multimediale kiosken te introduceren, waarbij de bezoeker via een tactiele interface een keuze kan maken tussen verschillende multimediafragmenten. Zo staat er bijvoorbeeld bij de sectie die handelt over de landing op de maan een kiosk die de gebruiker toelaat om naar geluidsfragmenten van de Apollo 11 bemanning te luisteren.

Hoewel de kiosken professioneel ogen en hun werk degelijk uitvoeren, zijn er enkele problemen met het huidige systeem. De multimediafragmenten bevinden zich namelijk op een Dvd en worden verwerkt door een Dvd-speler, waarbij de gebruikersinterface bestaat uit 4 grote knoppen intern doorverbonden met de afstandsbediening.
Het grote probleem met deze opzet is de levensduur: een Dvd-speler die continu actief is, verslijt zichzelf alsook de media die het afspeelt snel. Vervanging van de Dvd-speler is ook niet eenvoudig, daar het model dat indertijd aangekocht is niet meer in productie is, en nieuwe modellen niet altijd compatibel zijn met de afstandsbediening ingebouwd in de kiosk.
Ook zijn de mogelijkheden die het systeem biedt, sterk beperkt. Alle interactiviteit moet immers geïmplementeerd worden via Dvd-menu's, wat niet veel meer toe laat dan eenvoudige selectie van het multimediafragment.

Vandaar dit project, wat de interne naam \emph{Ad-Astra III} heeft meegekregen. Voorafgegaan door Ad-Astra I (vernieuwing van het telescopenpark) en II (vernieuwing van de tentoonstelling), zal het instaan voor de vernieuwing van de museumkiosken. Het uiterlijk zal hetzelfde blijven: een kiosk zal nog steeds gestuurd worden door 4 grote knoppen, alsook zal de weergave gerealiseerd worden op een (niet aanraakgevoelig) LCD scherm, eventueel uitgebreid met een set aan luidsprekers.
Intern zal het systeem echter volledig anders werken. Een duurzame en energiezuinige chip haalt de voorstellingen op van een centrale server, waardoor het makkelijker zal zijn wijzigingen aan te brengen en die ook direct weer te geven op de kiosken. Ook zal elke kiosk continu in verbinding staan met het netwerk, wat beter beheer alsook weergave van dynamisch materiaal toelaat. Tenslotte zullen de voorstellingen opgebouwd zijn in een flexibel framework, wat toelaat veel rijkere inhoud weer te geven.


%
% Structuur
%

\chapter{Structuur}
\label{chat:structuur}

Deze scriptie zal beginnen met een uiteenzetting over het algemeen ontwerp in deel \ref{ontwerp}: hoe wordt het systeem gemodelleerd, voor welke technologieën is er gekozen, welke eisen worden aan de hardware gesteld, enzovoort.

Vervolgens wordt de realisatie van elk van de deelsystemen uit de doeken gedaan: de server in deel \ref{server}, de kiosk in deel \ref{kiosk}, en de ontwikkeling van de interface module in \ref{inputmodule}.

Tenslotte wordt er aandacht besteed aan de effectieve invoering van het geheel in deel \ref{invoering}. Hierbij zullen we ook tijd besteden aan het testen van het systeem, alsook aan analyse van de prestaties van het systeem in dergelijke condities. Zo moeten bijvoorbeeld de kiosken blijven werken, ook als de primaire server onbereikbaar is.
