\part{Kiosk}
\label{kiosk}

\chapter{Structuur}
\label{kiosk:structuur}

De manier waarop we de kioskapplicatie hebben opgedeeld kent sterke gelijkenissen met de manier waarbij we de serverapplicatie gestructureerd hebben: alle logische componenten (de networkinterface, de userinterface en de datamanager) worden geïsoleerd in een apart subsysteem, en een overkoepelende controller staat in voor interacties die over verschillende subsystemen heen gaan.

\section{Netwerk subsysteem}
\label{kiosk:structuur:netwerk}

Dit gedeelte van de applicatie neemt alle netwerkcommunicatie op zich. Daartoe creëert het een \ac{upnp} device, registreert het de nodige services, en broadcast het die gegevens. Wanneer een bepaalde actie aangeroepen wordt, zal de respectievelijke service louter een signaal uitsturen. Dat signaal (bijvoorbeeld \code{reboot} of \code{setVolume(uint)}) zal vervolgens opgevangen worden door de applicatiecontroller, die dan tot de effectieve actie overgaat.

Door de effectieve implementatie in een andere klasse te plaatsen vereenvoudigen we de netwerkinterface die in de minimale opzet al complex genoeg is (beheren van \ac{upnp} state variables, conversie van parameters, registratie van \ac{scpd} bestanden, \ldots). Tevens wordt de hele component hierdoor platform-onafhankelijk, en moeten we bij het herimplementeren voor een nieuw platform enkel de implementatie herschrijven die volledig geïsoleerd is binnen de applicatiecontroller. Dit is opnieuw vergelijkbaar met de opzet die we gehanteerd hebben bij de serverapplicatie: code binnen een bepaald subsysteem staat enkel in voor verwerking relevant toe dat subsysteem, acties die ergens anders een impact hebben worden afgehandeld door de applicatiecontroller.

\subsection{Datamanager}
\label{kiosk:structuur:datamanager}

Deze klasse komt overeen met het repository subsysteem in de serverapplicatie, maar heeft een andere naam gekregen wegens een \emph{namespace clash} met enkele klassen die we gebruiken uit een library. De taken veranderen echter niet: de datamanager staat in  voor het ophalen van gegevens die zich in een externe \ac{svn} repository bevinden, en alles dat daarmee gepaard gaat. Zo zal de component moeten rekening houden met een eventuele cache, om zo een checkout van gegevens te versnellen. Ook moet het bij het opstarten kunnen controleren of er geen oude checkout aanwezig is, om zo direct al een voorstelling te kunnen weergeven, zelfs als dat een oude voorstelling betreft.

In tegenstelling tot de repository klasse in de serverapplicatie moeten we hier geen data interpreteren: na een checkout of update gedaan te hebben, moet de voorstelling in zijn geheel opnieuw doorgegeven worden aan de userinterface, het is niet van belang om te weten wélke gegevens veranderd zijn. Hierdoor wordt het gebruik van de \ac{svn} libraries sterk vereenvoudigt.

% backup cache uitleggen? config saven enzo

\subsection{Userinterface}
\label{kiosk:structuur:userinterface}

Dit gedeelte van de applicatie staat in voor de effectieve weergave van ontvangen voorstellingen. Zoals reeds vermeld bestaan die voorstellingen uit \ac{html} en Javascript, en gebruiken we de WebKit rendering engine om die gegevens weer te geven. In de huidige opzet hebben we enkel voorzien in een naadloze integratie van de rendering engine en onze applicatie, waardoor de voorstellingen zeer vlot geladen en verwerkt kunnen worden. Voor een volgende versie plannen we deze component uit te breiden zodat het niet alleen voorstellingen weergeeft, maar gegevens over hoe exact die weergegeven worden teruggestuurd worden naar de server. Zo kunnen we bijvoorbeeld statistieken vergaren over welke voorstellingen het meest bekeken worden, welke filmpjes het snelst gestopt worden, \ldots

Omdat de integratie van de rendering engine en de rest van de applicatie zo goed bleek te werken, hebben we besloten om de overige gebruikersinterfaces op eenzelfde manier te implementeren in \ac{html} en Javascript. Hoewel er niet zoveel reguliere interfaces binnen de applicatie te vinden zijn (opstart-pagina, en enkele debug pagina's), resulteerde dit in een uniforme userinterface-implementatie waarbij het zelf op termijn mogelijk zou moeten zijn om de interface code dynamisch te updaten op eenzelfde manier als dat bij de voorstellingen gebeurt.

Er is echter een groot verschil tussen een ordinaire voorstelling en een userinterface-pagina bij de applicatie: waar een voorstelling volledig \emph{self-contained} is, heeft een informatieve userinterface-pagina gegevens nodig uit de applicatie. Een debugpagina bijvoorbeeld heeft nood aan een mechanisme om debuggegevens van de applicatie te ontvangen.
Een mogelijkheid zou zijn om via \ac{ajax} gegevens op te halen die via een webservice lokaal opengesteld worden. Dit verhoogt echter de complexiteit van zowel de interface als de kioskapplicatie, laat staan dat het een efficiënte manier is om een weinig gegevens over te brengen. Daarom hebben we gekozen voor een secundaire piste, waarbij we gebruik maken van de \makeurl{http://doc.qt.nokia.com/latest/qtwebkit-bridge.html}{QtWebKit Bridge}. Dit mechanisme laat ons toe om een klasse binnen onze C++ applicatie toegankelijk te maken vanuit Javascript code, om zo gegevens op een efficiënte manier te kunnen doorspelen. Zo hebben we voor alle soorten pagina's die we moeten kunnen weergeven, een meta-klasse aangemaakt die de nodige gegevens aggregeert (zie fragment \ref{lst:expose_cpp}). Bij het aanmaken van de pagina in kwestie wordt die meta-klasse vervolgens geregistreerd binnen de rendering engine, waardoor de Javascript code die met de pagina geladen wordt er toegang tot heeft. Een voorbeeld van dit laatste is te zien in fragment \ref{lst:expose_js}.

\begin{lstlisting}[language=C++, float, caption=Registratie van een klase binnen de rendering engine., label=lst:expose_cpp]
class LogPage : public QWebPage
{
Q_OBJECT
Q_PROPERTY(QString id READ id CONSTANT)
public:
    LogPage(QObject* parent = 0);
    ~LogPage();

    QString id() const;
signals:
    void newMessage(const QString& iMessage);
};

LogPage::LogPage(QObject *parent) : QWebPage(parent)
{
    mainFrame()->addToJavaScriptWindowObject(
    	"application",
        this);
}
\end{lstlisting}

\begin{lstlisting}[language=JavaScript, float, caption=Gebruik van een geregistreerde C++ klasse., label=lst:expose_js]
// Gebruik van een signaal
function showMessage(message)
{
	...
}
application.newMessage.connect(showMessage);

// Gebruik van een property
document.write(application.id)
\end{lstlisting}


\chapter{Realisatie}
\label{kiosk:realisatie}

Tijdens de realisatie van de kioskapplicatie zijn we op enkele interessante problemen gestuit die zeker het vermelden waard zijn.

\section{Uniek ID}
\label{kiosk:realisatie:id}

Aangezien de software die op de kiosken zal draaien volledig identiek is -- we streven immers naar een configuratieloze opzet -- moet de applicatie in staat zijn om een identifier te genereren die uniek is, maar toch gebonden aan een specifieke kiosk om er vanuit de configuratie naar te kunnen verwijzen. De identifier zal gebruikt worden door de \ac{upnp} library, die het doorspeelt naar een externe partij als deel van de device description. Daarom zullen we ons moeten schikken naar de \ac{upnp} standaard, die een \ac{uuid} vereist als identifier.

Een \ac{uuid} is zeer specifiek opgemaakt, en bestaat steeds uit 16 bytes, onderverdeeld in 5 groepen die elk gescheiden zijn door een steepje: \code{DEADBEEF-E29B-41D4-A716-446655440000}. Er bestaan echter verschillende \ac{uuid} varianten, elk met een aantal versies. Zowel de variant als de versie wordt in de \ac{uuid} string geëncodeerd, met als template \code{xxxxxxxx-xxxx-Mxxx-Nxxx-xxxxxxxxxxxx}. De officiële specificatie detailleert maar 1 variant, die geïdentificeerd wordt door de twee meest-significante bits van de \code{N} byte op $1 0$ in te stellen. Betreffende de versie hebben we de keus uit een aantal versies, en kiezen we voor een gewijzigde vorm van de \ac{mac} versie. Hierbij moet de \code{M} byte ingesteld worden op een hexadecimale 1, en kunnen we de eerste twee groepen (samen exact 12 bytes) gebruiken om het \ac{mac} adres in te encoderen.

Normaal worden de overige bits vervolgens gevuld met tijdsgegevens, maar omdat we willen dat onze identifier identiek blijft na een reboot kiezen we ervoor om die bits op 0 in te stellen. Het resultaat hiervan valt te vinden in fragment \ref{lst:uuid}.

\begin{lstlisting}[language=C++, float, caption=Generatie van een \acs{uuid}., label=lst:uuid]
QUuid getHardwareUuid() const
{
  // Maak een interface request object aan
  struct ifreq tRequest;
  bzero(&tRequest, sizeof(tRequest));
  tRequest.ifr_addr.sa_family = AF_INET;
  strncpy(tRequest.ifr_name, "eth0", IFNAMSIZ-1);

  // Open een socket en voer de systeemaanroep uit
  int tFd = socket(AF_INET, SOCK_DGRAM, 0);
  ioctl(tFd, SIOCGIFHWADDR, &tRequest);
  close(tFd);

  // Genereer een UUID
  char* tMAC = ifr.ifr_hwaddr.sa_data
  QUuid oUuid;
  oUuid.data1    |= (unsigned char) tMAC[0] << 24;
  oUuid.data1    |= (unsigned char) tMAC[1] << 16;
  oUuid.data1    |= (unsigned char) tMAC[2] << 8;
  oUuid.data1    |= (unsigned char) tMAC[3];
  oUuid.data2    |= (unsigned char) tMAC[4] << 8;
  oUuid.data2    |= (unsigned char) tMAC[5];
  oUuid.data4[0]  = (oUuid.data4[0] & 0x3F  ) | 0x80;
  oUuid.data3     = (oUuid.data3    & 0x0FFF) | 0x1000;
  
  return oUuid;
}
\end{lstlisting}

\section{BRisa \acs{upnp} library}
\label{kiosk:realisatie:brisa}

Zoals reeds vermeld gebruiken we de BRisa library aan de kant van de kiosk, waarbij acties en methodes opgeroepen worden door de Cling bibliotheek aan de kant van de server. Cling staat er echter voor bekend om de \ac{upnp} specificatie op de letter te volgen, waarbij elke afwijking daarvan als een waarschuwing of zelfs vaak als een regelrechte fout beschouwd wordt. En jammer genoeg bleek de BRisa bibliotheek soms laks om te gaan met de standaarden voorgeschreven door de \ac{upnp} standaard \ldots

Om Cling te overtuigen om zonder veel probleem te werken met clients die de BRisa bibliotheek gebruiken, hebben we twee patches moeten doorvoeren. De eerste daarvan introduceert een extra spatie voor het \code{s:encodingStyle} attribuut dat meegestuurd wordt bij elk \ac{soap} bericht. Zonder deze spatie weigerde Cling de antwoorden van de BRisa bibliotheek te verwerken, waardoor het onmogelijk een actie uit te voeren. Een tweede patch voegt de optionele \code{content-type} header toe aan berichten waarbij de ontwikkelaars van BRisa dat vergeten waren, waardoor Cling nu ook zonder waarschuwingen kan communiceren met een BRisa server.

Hierbij was het een groot voordeel dat de library open-source was. Nadat we het probleem gelocaliseerd hadden door de foutmeldingen van Cling te combineren met een analyse van de verzonden berichten, konden we eenvoudig de code van de library ophalen, het probleem localiseren en oplossen. Hoewel we beide patches ingestuurt hebben op de bug tracker van het BRisa project (\makeurl{https://garage.maemo.org/tracker/index.php?func=detail\&aid=6953\&group\_id=138\&atid=583}{eerste patch}, \makeurl{https://garage.maemo.org/tracker/index.php?func=detail\&aid=6954\&group\_id=138\&atid=583}{tweede patch}), kon het gerust nog even duren vooraleer de wijzigingen toegepast worden, of dat een nieuwe versie zou vrijgegeven worden. Daarom hebben we een \makeurl{https://github.com/MIRAvzw/qt-brisa}{onze eigen versie} van het project gemaakt (ook wel een \emph{fork}) genoemd, waarin de fouten reeds gecorrigeerd zijn. Op de toestellen waarop de applicatie ontwikkeld en geïnstalleerd wordt kunnen we nu in afwachting van een nieuwe versie onze eigen versie van de bibliotheek op installeren.

\section{svnqt \acs{svn} library}
\label{kiosk:realisatie:svnqt}

Zoals beschreven in hoofdstuk \ref{ontwerp} hebben we voor onze kioskapplicatie een interessante \ac{svn} library gevonden die gebruik maakt en geïntegreerd is met het Qt framework dat we gebruiken. De code was oorspronkelijk een stand-alone library, genaamd ``svncpp" en ontwikkeld door RapidSVN in 2002. Nadien is de code geïntegreerd en verder ontwikkeld als deel van de KDESvn applicatie. Om de code eenvoudig doch efficiënt te kunnen gebruiken binnen onze kioskapplicatie hebben we code uit de KDESvn applicatie geëxtraheerd en er een \makeurl{https://github.com/MIRAvzw/svnqt}{alleenstaande bibliotheek} van gemaakt. Buiten enkele kleine wijzigingen hebben we niks moeten wijzigen om deze code vervolgens te kunnen gebruiken binnen ons project.

\section{Log4qt library}
\label{kiosk:realisatie:log4qt}

Log4qt is een project dat een Log4J-achtige interface aanbiedt om op eenvoudige manier een consistente logging te voorzien, waarbij het mogelijk is om via een configuratiebestand de specificeren waar de logberichten terechtkomen (zoals te zien is in fragment \ref{lst:log4qt}). Hoewel de bibliotheek functioneel vrij compleet is en op het eerste zicht geen problemen opleverde, hebben we naar verloop van tijd toch enkele wijzigingen toegebracht aan de \emph{upstream} code.

\begin{lstlisting}[language=JavaProperties, float, caption=Externe configuratie van Log4qt., label=lst:log4qt]
# Registreer alle appenders
log4j.rootLogger          = DEBUG, dbg

# Debug appender
log4j.appender.dbg        = org.apache.log4j.FileAppender
log4j.appender.dbg.file   = debug.log
log4j.appender.dbg.layout = org.apache.log4j.TTCCLayout
\end{lstlisting}

Een \makeurl{https://gitorious.org/log4qt/log4qt/merge\_requests/3}{eerste patch} die we ingestuurd hebben corrigeerde een probleem met namespaces. C++ code die gebruik maakt van Qt wordt immers eerst gepreprocessed door een Qt-specifieke preprocessor: de \ac{moc}. Deze preprocessor zorgt ervoor dat Qt meta-object en introspectiemogelijkheden krijgt, wat krachtige mechanismen zoals \code{signals \& slots} mogelijk maakt. Dit mechanisme is echter niet altijd even robuust: externe factoren kunnen ertoe leiden dat een \ac{moc}-macro verkeerd expandeert, en dat was hier het geval bij een ontbrekende namespace prefix binnen een \code{Q\_PROPERTY} \ac{moc}-macro.

Vervolgens hebben we ook enkele kleine features toegevoegd die het werken met de bibliotheek aangenamer maken. Zo hebben we voorzien in een \makeurl{https://gitorious.org/log4qt/log4qt/merge\_requests/4}{patch} die het mogelijk maakt om de Log4qt objecten te gebruiken met output operatoren: dit is een veelgebruikte techniek in de officiële Qt libraries waardoor de Log4qt bibliotheek eenvoudiger is voor mensen die er nog niet thuis in zijn maar toch ervaring hebben met Qt. Op vergelijkbare wijze hebben we de manier om headers te includen \makeurl{https://gitorious.org/log4qt/log4qt/merge\_requests/5}{gewijzigd} zodat het eveneens dichter aanleunt bij de manier waarop dat gebeurt bij de officiële Qt code.

\section{Statische codeanalyse}
\label{kiosk:realisatie:codeanalyse}

Net zoals we dat bij de serverapplicatie gedaan hebben, willen we met behulp van statische codeanalyse proberen de kwaliteit van de code te verbeteren. Jammer genoeg blijkt het voor C++ code veel moeilijker te zijn om aan statische analyse te doen. Het grote probleem is dat van de \emph{preprocessor}: door toe te laten dat macro's op een heel flexibele manier arbitraire code kunnen genereren is het veel moeilijker om eenduidig op zoek te gaan naar fouten in die code. Zeker in het geval van Qt, dat sterk berust op macro's om geavanceerde technieken zoals signalen en introspectie te implementeren.

\subsection{Stijl}
\label{kiosk:realisatie:codeanalyse:stijl}

Na lang gezocht te hebben bleek er geen goede tool te bestaan om de stijl van een C++ codebase te controleren, wat we bij de serverapplicatie wel hebben kunnen doen met Checkstyle. Het beste dat we gevonden hebben is \makeurl{https://google-styleguide.googlecode.com/svn/trunk/cpplint/cpplint.py}{cpplint}, een script door Google ontworpen dat ze intern gebruiken om hun C++ code te standaardiseren. Maar zoals nogal snel duidelijk werd is het een veel minder krachtige applicatie: het voert enkel wat rudimentaire controles uit op vlak van indentatie en andere elementaire syntax, geavanceerde modules zoals Checkstyle ze kent vinden we hier niet terug. Gelukkig is de kioskapplicatie een pak minder complex dan de serverapplicatie, waardoor het toch nog haalbaar is om de code af en toe manueel te overlopen.

\subsection{Correctheid}
\label{kiosk:realisatie:codeanalyse:correctheid}

Om de correctheid van C++ code te analyseren bestaan er al wat meer tools, zoals het veelgebruikte en relatief uitgebreide \makeurl{http://cppcheck.sourceforge.net/}{cppcheck}. Deze tool controleert onder andere het gebruik van arrays, of excepties correct afgehandeld worden, en of er geen grote memory leaks zijn. Dankzij deze applicatie hebben we verschillende codefragmenten kunnen identificeren waar objecten onnodig gekopiëerd werden, wat het geheugengebruik ten goede gekomen is.

Een probleem met cppcheck is dat het niet goed overweg kan met de vele macro's specifiek aan Qt. Daarom zijn we op zoek gegaan naar een statische codeanalysetool die op zijn minst geen problemen heeft met de constructies die we terugvinden bij Qt applicaties. En zo een tool hebben we ook gevonden, onder de naam \makeurl{http://www.englishbreakfastnetwork.org/krazy/}{Krazy}. Deze applicatie is deel van het \ac{kde} project, een desktop environment die hoofdzakelijk bestaat uit Qt en C++. Krazy is dan ook specifiek gemaakt om niet alleen overweg te kunnen met Qt code, maar om ze ook expliciet te controleren op fouten. Hoewel Krazy veel onbelangrijke problemen vermeldde (die vaak hun oorsprong kennen in de \ac{kde} coding policies), heeft het ook effectief een serieuze fout gevonden. Bij een bepaalde klasse hadden we immers een essentiële macro vergeten die absoluut nodig is om een correct object te verkrijgen.

\chapter{Deployment}
\label{kiosk:deployment}

\section{Besturingssysteem}
\label{kiosk:deployment:besturingssysteem}

Zoals vermeld in \ref{server:deployment:besturingssysteem} was het voor de serverapplicatie niet haalbaar om te switchen naar een krachtiger besturingssysteem wegens verschillende hardwarebeperkingen. Bij de meer \emph{general-purpose} hardware die we voor de kiosken gekocht hebben is dit anders: het toestel komt met een \ac{sd} kaartslot, en het is met de standaard bootloader mogelijk om zo op te starten van een besturingssysteem dat zich op een \ac{sd} kaart bevindt. Hierdoor kunnen we veel gemakkelijker een alternatief besturingssysteem ontwikkelen, omdat er nu snel gedeployed kan worden naar een \ac{sd} kaartje dat we dan ook zonder gevaar op beschadiging direct kunnen testen op het toestel.

% todo

\section{Versioning}
\label{kiosk:deployment:versioning}

Ook voor de kioskapplicatie maken we gebruik van het Git versioning systeem, wat we in sectie \ref{server:deployment:versioning} beschreven hebben is dus hier ook van toepassing.

\section{Compilatie}
\label{kiosk:deployment:compilatie}

Om de software te compileren maken we gebruik van de tools die het Qt framework daartoe biedt. De belangrijkste daarvan is \code{qmake}, een applicatie die het genereren van een \code{Makefile} automatiseert. Dit is vooral interessant omdat zoals hierboven vermeld bronbestanden die gebruik maken van Qt-macro's eerst moeten voorverwerkt worden door de \ac{moc} preprocessor. Door gebruik te maken van qmake moeten we dit niet langer manueel doen: qmake detecteert welke bestanden eerst moeten verwerkt worden door de preprocessor, en genereert een \code{Makefile} waarbij dit dan ook eerst gebeurt.

Ook hebben we tijdens de ontwikkeling van de applicatie gebruik gemaakt van \makeurl{http://clang.llvm.org/}{Clang}, een moderne compiler voor C en C++. Hoewel deze vrij jonge compiler nog niet op hetzelfde niveau staat als de \ac{gcc}, is het op enkele vlakken veel beter dan \ac{gcc}. Zo is er zeer veel aandacht besteed aan het error reporting subsysteem, waardoor de foutmeldingen die Clang genereert vaak een pak behulpzamer zijn dat wat \ac{gcc} in eenzelfde situatie zou laten weten. Toch worden de uiteindelijke executables gegenereerd met \ac{gcc}, omdat die op vlak van optimalisaties (zowel op vlak van snelheid als grootte) nog steeds de beste keuze is.

Tenslotte is het nog het vermelden waard dat alle libraries die we gebruiken dynamisch gelinkt worden met de finale executable. Dit betekent dat de libraries op het systeem aanwezig moeten zijn, en niet met de applicatie gebundeld worden. Bij de serverapplicatie hebben we dit net anders gedaan, zoals beschreven in sectie \ref{server:deployment:compilatie} hebben we daar alles gebundeld tot een enkel monilitisch geheel. De redenering hierachter is tweeledig. Enerzijds is een kiosk veel statischer: eenmaal we het besturingssysteem samengesteld hebben en er alle nodige applicaties op geïnstalleerd hebben zullen we er lange tijd af blijven. Bij de server is dat anders, niet alleen kunnen en zullen we vaker van hardware veranderen (waarbij vervolgens het besturingssysteem vernieuwd moet worden), ook kunnen we het ons bij een genetwerkte server uit veiligheidsoverwegingen niet veroorloven om jaren aan een stuk dezelfde software te gebruiken. Anderzijds verschillende ook de beschikbare voorzieningen om code in gedeelde bibliotheken af te zonderen: bij C++ code is dit bijvoorbeeld heel eenvoudig dankzij de ingebouwde en sterk geïntegreerde \emph{dynamic linker}. Voor Java applicaties is dit niet het geval: archieven kunnen op arbitraire plaatsen terechtkomen en er zijn geen algemeen aanvaarde mechanismen om die locaties eenduidig te ontdekken en op te vragen bij het opstarten van een applicatie.
