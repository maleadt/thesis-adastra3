\part{Server}
\label{server}

\chapter{Structuur}

Op het hoogste niveau hebben we de serverapplicatie onderverdeeld in enkele logische subsystemen, zijnde het netwerk en de repository. Onderling communiceren deze subsystemen niet met elkaar: coördinatie verloopt door een overkoepelende controller. Zo is het mogelijk om elk subsysteem als een alleenstaand geheel te behandelen, wat het ontwikkelingsproces sterk vereenvoudigt. Tenslotte is er nog de webinterface, die dezelfde visibiliteit heeft als de controller: acties die een gebruiker uitvoert op de beheersinterface kunnen een impact hebben op objecten binnen het netwerk of de repository.

Om code te abstraheren en controle van subsystemen te vereenvoudigen moet elk van de subsystemen een bepaalde klasse uitbreiden met de concrete implementatie. Die klasse voorziet niet alleen in bepaalde gedeelde functionaliteit (zoals het genereren van log berichten, of inladen van de nodige configuratiebestanden), maar schrijft ook voor hoe de effectieve implementatie moet gevormd worden. Hierdoor wordt het eenvoudig om vanuit de applicatiecontroller op een eenduidige manier bepaalde standaardacties (zoals het opstarten of afsluiten van een subsysteem) uit te voeren.

\section{Netwerk subsysteem}

Het netwerk subsysteem verzorgt wat we in de vorige hoofdstukken tot applicatieprotocol gedoopt hebben. Het subsysteem is opgesplitst in enkele samenwerkende componenten: vooreerst is er het gedeelte dat dient als main entrypoint voor andere componenten die willen gebruik maken van het netwerksubsysteem. Daartoe biedt het bijvoorbeeld een lijst van aanwezige kiosken, en kan voor elke kiosk een object opgevraagd worden dat toelaat om gegevens op te halen of acties uit te voeren. Ook is er voorzien in de nodige functionaliteit om signalen door te geven, bijvoorbeeld om geschikt te reageren op de toevoeging van een nieuwe kiosk.
Door al deze functionaliteit te bundelen in een enkele klassie, is dit de enige interface tussen het volledige subsysteem en de rest van de applicaties. De uiteindelijke implementatie van het subsysteem kan hierdoor eenvoudig gewijzigd worden, zonder veel te moeten veranderen aan componenten die gebruik maken van dit systeem.

Het netwerk subsysteem bevat ook nog een gedeelte dat instaat voor de effectieve monitoring van het netwerk, door te reageren op signalen van de achterliggende \ac{upnp} bibliotheek. Dit gedeelte is puur intern, en het is niet de bedoeling dat een externe component communiceert met deze netwerk monitor. Als de monitor bepaalde informatie wil vrijgeven aan de buitenwereld (wat bijvoorbeeld voorkomt als het een nieuwe kiosk detecteerd), zal het dit doorgeven aan de publieke component van het netwerk subsysteem, waarna die bijvoorbeeld bepaalde signalen zal uitsturen. Hoewel dit geconvolueerd mag lijken is dit een hele interessant design: zoals hierboven vermeld mogen we nu steeds de monitor naar believen aanpassen, zolang de publieke interface maar dezelfde blijft zal niemand hier iets merken.

Om het voor de buitenwereld eenvoudig te maken om te interfacen met het netwerk, zal de publieke interface niet alleen informatie vrijgeven maar ook toegang bieden tot functionele objecten die eenvoudige manipulatie van hun onderwerp toelaten. Zo zal een \code{Network::getDevice(String uuid)} aanroep niet louter informatie over het toestel teruggeven, maar een volwaardig \code{Device} object. Dit object kan dan gebruikt worden om het toestel zelf te manipuleren: indien we bijvoorbeeld over een kiosk spreken zal eer voorzien zijn van een \code{Device::Shutdown} methode.

\section{Repository subsysteem}

Dit systeem dient als wrapper rond de repository waarbinnen alle gegevens van het systeem (configuraties en voorstellingen) opgeslagen zijn. Zoals vermeld in hoofdstuk \ref{ontwerp} staat de server in voor het ophalen en verwerken van de configuratiegegevens, en moet het de nodige kiosken laten weten wanneer de media die ze tonen gewijzigd is.

Net zoals bij het netwerk subsysteem biedt het repository subsysteem een publieke interface die de buitenwereld toelaat gegevens op te vragen. Ook zullen de teruggegeven objecten opnieuw gesofisticeerd zijn om eenvoud verwerken van de gegevens toe te laten: de \code{Repository::getConfiguration(String id)} aanroep bijvoorbeld zal daarom een \code{KioskConfiguration} object teruggeven dewelke specifieke functies bevat zoals \code{getVolume}. Hoewel dit systeem zijn nadelen kent -- een wijziging van het configuratieformaat vereist direct wijzigingen aan de servercode -- zorgt de rigide opzet ervoor dat inconsistenties tussen de repository en de achterliggende logica snel zullen duidelijk worden. 

Bij het opstarten van het repository subsysteem zal een initiële \code{checkout} uitgevoerd worden. Hierna zal een interne timer gestart worden die om de minuut controleert of er geen nieuwe revisie in de repository te vinden is. Indien dat het geval is, zullen alle configuraties opnieuw binnengehaald worden om dan eventueel de wijzigingen naar een client te sturen. Momenteel gebeurt dit echter vrij inefficiënt: alle configuraties worden opnieuw opgehaald en verstuurd, los van het feit of er ook effectief wijzigingen gebeurd zijn. Voor een volgende versie streven we ernaar om ook effectief de delta-gegevens te interpreteren en zo intelligent de nodige veranderingen door te sturen.

\section{Applicatiecontroller}

Dit systeem staat in voor de coördinatie van acties die niet geïsoleerd blijven tot een enkel subsysteem. Het beste voorbeeld hiervoor is voor wanneer het netwerk subsysteem een nieuwe component detecteert. Om de code overzichtelijk te houden zal het netwerk subsysteem dan geen actie ondernemen, maar louter zijn gegevens updaten en een signaal uitsturen. De applicatiecontroller vangt vervolgens dit signaal op om vervolgens de repository te raadplegen voor eventuele configuratiegegevens.

\section{Website subsysteem}

Deze component visualiseert de gegevens uit de repository en netwerk subsystemen, en laat de gebruiker toe om er controle over uit te oefenen. Die controle is echter beperkt: persistente configuratie moet nog altijd verlopen via manuele interventie.

\chapter{Realisatie}

\section{Statische codeanalyse}

Om de kwaliteit van de code systematisch op te drijven, hebben we gebruik gemaakt van statische codeanalyse. Daarbij analyseert een applicatie de code zonder die ook effectief uit te voeren, om vervolgens een verslag te kunnen opbouwen waarbij de inhoud verschilt van applicatie tot applicatie.

\subsection{Stijl}

Een eerste soort van statische codeanalyse-applicaties toetst de codebase aan een verzameling regels om zo een uniforme codestijl te bekomen. Dit lijkt misschien niet zo nuttig, maar leesbare code en een uniforme codestijl maakt het veel eenvoudiger om de applicatie te onderhouden.
Voor Java bestaat er zo een heel krachtige applicatie, \makeurl{http://checkstyle.sourceforge.net/}{Checkstyle}, die geleverd wordt met een enorm uitgebreide set aan modules die elk een bepaald aspect analyseren. Om nog verfijnder tewerk te gaan kunnen kunnen we elke module vaak beperken met een aantal filters, zodat we bijvoorbeeld eenvoudig kunnen stellen dat de publieke methodes van een interface voorzien moeten zijn van Javadoc.

De standaardconfiguratie van Checkstyle is vrij strikt, waardoor een eerste controle van de serverapplicatie meer dan 1500 waarschuwingen opleverde. Daarom hebben we een eigen configuratie aangemaakt, gebaseerd op de standaardconfiguratie met de strengste eisen (zoals het beperken van de lijnlengte, of het verbieden van elk teveel aan spaties) wat versoepeld. Ook hebben we enkele eigen regels toegevoegd om praktijken die we reeds uitvoerden nu ook systematisch te kunnen controleren. Zo hebben we bijvoorbeeld een regel toegevoegd die bepaalt dat variabelen binnen een klasse geprefixt moeten worden met een kleine ``m'', parameters van een functie een ``i'' voor hun naam moeten krijgen, en lokale variabelen altijd met een ``t'' moeten beginnen. De configuratie van die ene regel is te zien in fragment \ref{lst:checkstyle:variabelen}.

\begin{lstlisting}[language=XML, float, caption="Checkstyle configuratie voor de naamgeving van variabelen., label=lst:checkstyle:variabelen]
<module name="LocalVariableName">
  <property name="format" value="^t[A-Z][a-zA-Z0-9]*$"/>
</module>
<module name="ParameterName">
  <property name="format" value="^i[A-Z][a-zA-Z0-9]*$"/>
</module>
<module name="MemberName">
  <property name="format" value="^m[A-Z][a-zA-Z0-9]*$"/>
</module>
\end{lstlisting}

\subsection{Correctheid}

Dergelijke applicaties gaan op zoek naar logische fouten in de code, waarbij de code wel legaal is maar ze bijvoorbeeld de applicatie doet crashen wanneer ze uitgevoerd wordt. Het is duidelijk dat dit een zeer complexe definitie is, die veel gesoftisticeerdere regels en heuristieken vereist dan wanneer men louter de codestijl wil controleren. Het probleem met zulke applicaties is dan ook dat er vaak een lage \emph{signal to noise} verhouding is: of er worden veel valse positieven gedetecteerd, of veel fouten worden helemaal niet niet gevonden.

Om toch zoveel mogelijk fouten te vinden via statische analyse, hebben we gebruik gemaakt van meerdere tools. Vooreerst hebben we gekeken naar \makeurl{http://findbugs.sourceforge.net/}{FindBugs}, een tool die zoals de naam laat uitschijnen vooral op zoek gaat naar echte fouten. Hoewel de \makeurl{http://findbugs.sourceforge.net/bugDescriptions.html}{databank aan regels} die het daarbij gebruikt vrij uitgebreid is, had FindBugs niet veel te zeggen over onze code. We hopen dat dit een teken is dat de codebase van goede kwaliteit is.

Een andere applicatie die we gebruikt hebben is \makeurl{http://pmd.sourceforge.net/}{PMD}. Deze gaat niet alleen op zoek naar bugs, maar ook naar verkeerd gebruik van geheugen, overdreven complexe codefragmenten, suboptimale code, en code die nooit uitgevoerd wordt (\emph{dead code}). Zo hebben we met behulp van PMD verschillende suboptimale \emph{inner loops} kunnen versnellen, alsook enkele complexe klassen vereenvoudigt. Kritieke fouten zijn er echter opnieuw niet gevonden.

\chapter{Deployment}

\section{Besturingssysteem}

%todo

\section{Compilatie}

% todo: integratie statische codenalyse

Voor het bouwen van een executable hebben we berust op \ac{maven}, een softwarebeheersysteem dat op een flexibele manier toelaat om alle aspecten van het bundelen van een applicatie te organiseren. Daartoe voorzien we in een \ac{pom} bestant waarin alle informatie verwerkt zit, zoals de naam en versie van de toepassing, de libraries waarop het berust, en hoe de applicatie moet opgestart worden. Aan de hand van die informatie kan \ac{maven} het project gepast uitvoeren of bundelen tot een finale executable.

Ook wordt het gebruik van externe libraries vereenvoudigt: er bestaat een \ac{maven} repository met daarin de meest populaire libraries, waardoor we louter de naam en versie in het \ac{pom} bestand moeten specificeren en \ac{maven} met deze informatie zelf op zoek gaat naar de gepaste binaries die benodigd zijn om de applicatie op te starten (zie bijvoorbeeld fragment \ref{lst:maven}).

\begin{lstlisting}[language=XML, float, caption=Inladen van externe libraries via Maven., label=lst:maven]
<dependencies>
  <dependency>
    <groupId>log4j</groupId>
    <artifactId>log4j</artifactId>
    <version>1.2.16</version>
  </dependency>
  ...
</dependencies>
\end{lstlisting}

Voor enkele libraries die we niet terugvonden in de officiële \ac{maven} repositories hebben we ofwel een alternatieve repository toegevoegd die het pakket wel bevatte (zoals dat was in het geval van Cling), of zelf voorzien in een lokale repository die de benodigde bestanden bevat en gebundeld is met de broncode om iedereen toe te laten de applicatie zonder al te veel moeite te compileren.

Ook kent \ac{maven} een plugin-systeem, dat we dankbaar gebruiken om bepaalde specifieke taken uit te voeren. Zo willen we bijvoorbeeld een \ac{jar} bestand genereren dat tegelijk alle afhankelijke libraries bevat, om zo installatie op de uiteindelijke server te vereenvoudigen. Dit druist deels in tegen de ideale manier om software te packagen, waarbij externe libraries steeds via de lokale package-manager geïnstalleerd worden, maar omdat we daarvoor zeer nauwgezet het pakket zouden moeten onderhouden (updates van een externe library kan immers steeds problemen veroorzaken in de applicatie zelf) hebben we ervoor gekozen om alles tesamen te bundelen tot een monolitisch geheel. Specifiek voor de serverapplicatie is de relevante configuratie daarvoor te vinden in fragment \ref{lst:fatjar}.

\begin{lstlisting}[language=XML, float, caption=Gebruik van Maven modules om een executable te compileren., label=lst:fatjar]
<build>
  <plugins>
    <plugin>
    <groupId>org.apache.maven.plugins</groupId>
    <artifactId>maven-assembly-plugin</artifactId>
    <executions>
      <execution>
        <id>package-jar-with-dependencies</id>
        <phase>package</phase>
        <goals>
          <goal>single</goal>
        </goals>
        <configuration>
          <appendAssemblyId>false</appendAssemblyId>
          <descriptorRefs>
            <descriptorRef>
              jar-with-dependencies
            </descriptorRef>
          </descriptorRefs>
          <archive>
            <manifest>
              <mainClass>
                be.mira.adastra3.server.Main
              </mainClass>
            </manifest>
          </archive>
        </configuration>
      </execution>
    </executions>
  </plugin>
</build>
\end{lstlisting}

\section{Packaging}

% todo: debian packaging

\section{Webserver}

Zoals uitvoerig beschreven in hoofdstuk \ref{ontwerp} hebben we wegens het gebrek aan een serverside \ac{svn} library voor de server nood aan een externe \ac{svn} instantie. Om het beheer van zowel de serverapplicatie als die \ac{svn} instantie eenvoudig te houden, gaan we beide overkoepelen door een Apache webserver.

De eerste daarbij is de configuratie van een \ac{svn} instantie. Hierbij is het van belang dat enkel een beheerder die data kan wijzigen, maar iedereen er toegang tot heeft (er is geen nood aan authenticatie, de voorstellingen bevatten geen gevoelige data). Na het aanmaken van een \ac{svn} repository op de server kunnen we die overkoepelen door de Apache webserver via de configuratie uit fragment \ref{lst:apache:svn}.

\begin{lstlisting}[language=ApacheConfig, caption=Configuratie van Apache om een \acs{svn} server te overkoepelen., label=lst:apache:svn]
<Location "/repository">
  DAV svn
  SVNPath /var/db/subversion
  AuthType Basic
  AuthName "Ad-Astra III - Repository"
  AuthUserFile /etc/httpd/conf/repository.htpasswd
  Order deny,allow
  <LimitExcept GET PROPFIND OPTIONS REPORT>
    Require valid-user
  </LimitExcept>
</Location>
\end{lstlisting}

Vervolgens moeten we nog de beheersinterface van de serverapplicatie overkoepelen. Hierbij configureren we de embedded Tomcat instantie eerst om toegankelijk te zijn via het \ac{ajp} protocol, wat we op eenvoudige manier mogelijk gemaakt hebben via het standaard configuratiebestand dat met de broncode van de serverapplicatie geleverd wordt. Daarnaast moeten we er Apache op wijzen dat requests op een bepaalde locatie nog moeten doorgestuurd worden naar een externe applicatie. Dit gebeurt via de Apache Tomcat connector, te vinden in \code{mod\_jk}. Zoals te zien is in de Apache configuratie van fragment \ref{lst:apache:jakarta} wordt daarbij nog verwezen naar een extern bestand: dit bevat de configuratie die \code{mod\_jk} nodig heeft om de externe applicatie bereiken, namelijk de locatie van de applicatie en hoe ermee moet gecommuniceerd worden (zie fragment \ref{lst:jakarta}).

\begin{lstlisting}[language=ApacheConfig, float, caption=Configuratie van Apache om een externe Jakarta applicatie te overkoepelen., label=lst:apache:jakarta]
// Inladen van de connector
LoadModule jk_module modules/mod_jk.so

// Specificeren van het configuratiebestand
JkWorkersFile /etc/httpd/conf/jakarta.properties

// Instellen van de logging
JkShmFile        /var/log/httpd/mod_jk.shm
JkLogFile        /var/log/httpd/mod_jk.log
JkLogLevel       info
JkLogStampFormat "[%a %b %d %H:%M:%S %Y] "

// Doorverwijzen van een bepaalde locatie
JkMount /status/* worker1
\end{lstlisting}

\begin{lstlisting}[language=JavaProperties, float, caption=Configuratie van de Apache Tomcat connector., label=lst:jakarta]
# Registreren van alle workers
worker.list=worker1

# Configureren van een worker
worker.worker1.type=ajp13
worker.worker1.host=localhost
worker.worker1.port=8009
\end{lstlisting}