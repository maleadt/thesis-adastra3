\part{Ontwerp}
\label{part:ontwerp}

%
% Systeemmodel
%

\chapter{Systeemmodel}
\label{chap:systeemmodel}

De eerste stap van het ontwerp was de identificatie van de verschillende deelsystemen. Hiertoe hebben we eerst gekeken naar de verschillende taken die het systeem als een geheel moet vervullen. Zo zijn er natuurlijk de kiosken, die instaan voor het weergeven van de presentaties, en het verwerken van gebruikersinput. Om het systeem flexibel te houden, zullen we de kiosken zo inrichten dat zowel de configuratie als de weer te geven presentaties zich niet op voorhand op de kiosk bevinden, maar van een centrale server gehaald worden. Diezelfde centrale server kan dan ook voorzien in een beheersinterface, waarbij de status van de verschillende kiosken gevisualiseerd wordt, en de administrator eventueel bepaalde acties kan ondernemen.

\begin{figure}
	\includegraphics[width=\textwidth]{diagrammen/ontplooiingsdiagram}
	\caption{Ontplooiingsdiagram}
\end{figure}

%
% Applicatie
%

\chapter{Applicatie}
\label{chap:applicatie}


%
% Monitoring
%

\chapter{Monitoring}
\label{chap:monitoring}


%
% Hardware
%

\chapter{Hardware}
\label{chap:hardware}
