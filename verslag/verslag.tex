%
% Configuratie
%

% Preambule met standaardinstellingen
\documentclass[a4paper,oneside,11pt,final]{memoir}

% Noot: zorg ervoor dat Nederlandse woord-splitsing geactiveerd is.
\usepackage[dutch]{babel}

% UTF8 gebruiken voor gebruik van alle symbolen
\usepackage[utf8]{inputenc}
\usepackage{eurosym}

% Tabellen eleganter maken
\usepackage{booktabs}

% Navigeerbaarheid van hyperlinks in PDF
\usepackage{hyperref}

% Emp voor MetaUML
\usepackage{emp}
\ifx\pdftexversion\undefined
\usepackage[dvips]{graphicx}
\usepackage[dvips]{iiiscriptie}
\else
\usepackage[pdftex]{graphicx}
\usepackage[pdftex]{iiiscriptie}
\DeclareGraphicsRule{*}{mps}{*}{}
\fi

% Noot: je kan het graphicxpakket een optie dvips of pdftex doorgeven
% in dat geval moet je ze ook aan iiiscriptie doorgeven, dus bijvoorbeeld
\usepackage{iiiscriptie}

% Extra functies
% Verkleinde margin entry
\setlength{\marginparwidth}{1.2in}
\let\oldmarginpar\marginpar
\renewcommand\marginpar[1] {\-\oldmarginpar[\raggedleft\footnotesize #1]%
{\raggedright\footnotesize #1}}

% Een TODO-entry
\newcommand{\todo}[1] {
	\addcontentsline{tdo}{todo}{\protect{#1}}
	\marginpar{#1}
}

% Hyperlink maken en URL in footnote tonen
\usepackage{hyperref}
\newcommand{\makeurl}[2]{\href{#1}{#2} \footnote{#1}}

% Compacte enumeraties
\newenvironment{enumerate_compact}{
\begin{enumerate}
  \setlength{\itemsep}{1pt}
  \setlength{\parskip}{0pt}
  \setlength{\parsep}{0pt}
}{\end{enumerate}}
\newenvironment{itemize_compact}{
\begin{itemize}
  \setlength{\itemsep}{1pt}
  \setlength{\parskip}{0pt}
  \setlength{\parsep}{0pt}
}{\end{itemize}}

% Float voor codefragmenten
\usepackage{float}
\floatstyle{ruled}
\newfloat{code}{thp}{lop}
\floatname{code}{Codefragment}


%
% Titelpagina
%

% Invullen velden
\departement{Departement Toegepaste Ingenieurswetenschappen}
\deptadres{Schoonmeersstraat 52 - 9000 Gent}
\studiejaar{1e Master Informatica}
\soortrapport{Scriptie voorgelegd tot het behalen van de titel van Master in de Toegepaste Ingenieurswetenschappen}
\title{Ontwikkeling van een museum kiosk}
\author{Tim BESARD}


%
% Inhoud
%

\begin{document}
\begin{empfile}
\begin{empcmds}
input metauml;
\end{empcmds}

%
% Titelpagina
%

\maketitle
\pagenumbering{roman}


%
% Abstract
%

\chapter*{Abstract}
\addcontentsline{toc}{chapter}{Abstract}
\label{chap:abstract}

%
% Voorwoord
%

\chapter*{Voorwoord}
\addcontentsline{toc}{chapter}{Voorwoord}
\label{chap:voorwoord}


%
% Inhoudstafel
%

\setlength\cftpartnumwidth{2em}

\newpage

\label{chap:inhoudstafel}
\tableofcontents

\newpage

\pagenumbering{arabic}



\part{Introductie}
\label{part:introductie}


%
% Doelpubliek
%

\chapter{Doelpubliek}
\label{chap:doelpubliek}


%
% Motivatie
%

\chapter{Situatie}
\label{chap:situatie}

De volkssterrenwacht MIRA biedt aan zijn bezoekers een uitgebreide rondleiding om die wat bij te leren op vlak van astronomie. Aangezien het overgrote deel van de bezoekers bestaat uit kinderen (de sterrenwacht wordt vaak bezocht in schoolverband, of door families), doet men sinds jaar en dag moeite om de rondleiding zo interessant mogelijk te houden, ook wanneer er geen gids is om in te staan voor interactiviteit. Daartoe heeft men enkele jaren geleden een project gelanceerd, genaamd \emph{Ad-Astra}. Het doel van dit project was om multimediale kiosken te introduceren, waarbij de bezoeker via een tactiele interface een keuze kan maken tussen verschillende multimediafragmenten. Zo staat er bijvoorbeeld bij de sectie die handelt over de landing op de maan een kiosk die de gebruiker toelaat om naar geluidsfragmenten van de Apollo 11 bemanning te luisteren.

Hoewel de kiosken professioneel ogen en hun werk degelijk uitvoeren, zijn er enkele problemen met het huidige systeem. De multimediafragmenten worden immers door een Dvd-speler ingeladen vanop een Dvd, waarbij de gebruikersinteractie gerealiseerd wordt door 4 grote knoppen op de kiosk intern door te verbinden met de afstandsbediening.
Het grote probleem met deze opzet is de levensduur: een Dvd-speler die continu actief is, verslijt zichzelf alsook de media die het afspeelt vrij snel. Vervanging van de Dvd-speler is ook niet eenvoudig, daar het model dat indertijd aangekocht is niet meer in productie is, en nieuwe modellen niet altijd compatibel zijn met de afstandsbediening ingebouwd in de kiosk.
Ook zijn de mogelijkheden die het systeem biedt, sterk beperkt. Alles moet immers geïmplementeerd worden als een Dvd-menu, wat niet veel meer toe laat dan selectie van het multimediafragment.

Vandaar dit project, intern gekend als het \emph{Ad-Astra III} project, dat een systeem zal ontwikkelen ter vervanging van de huidige implementatie. Het uiterlijk zal hetzelfde blijven: een kiosk zal nog steeds gestuurd worden door 4 grote knoppen, alsook zal de weergave gerealiseerd worden op een (niet aanraakgevoelig) LCD scherm, eventueel uitgebreid met een set aan luidsprekers.
Intern zal het systeem echter volledig anders werken. Gebaseerd op een duurzame en energiezuinige chip zal de presentatie afgehaald worden van een centrale server, waardoor het makkelijker zal zijn wijzigingen aan te brengen en die ook direct weer te geven op de kiosken. Ook zal elke kiosk continu in verbinding staan met het netwerk, wat beter beheer alsook weergave van dynamisch materiaal toelaat. Tenslotte zullen de presentaties opgebouwd zijn in een flexibel raamwerk, wat toelaat veel rijkere inhoud weer te geven.


%
% Structuur
%

\chapter{Structuur}
\label{chat:structuur}

Deze scriptie zal beginnen met een uiteenzetting over het algemeen ontwerp in deel \ref{part:ontwerp}: hoe wordt het systeem gemodelleerd, voor welke technologieën is er gekozen, welke eisen worden aan de hardware gesteld, enzovoort.

Vervolgens wordt de realisatie van elk van de deelsystemen uit de doeken gedaan: de server in deel \ref{part:server}, en de kiosk in deel \ref{part:kiosk}.

Tenslotte wordt er aandacht besteed aan de effectieve invoering van het geheel in deel \ref{part:invoering}. Hierbij zullen we ook tijd besteden aan het testen van het systeem, alsook aan analyse van de prestaties van het systeem in dergelijke condities. Zo moeten bijvoorbeeld de kiosken blijven werken, ook als de primaire server onbereikbaar is.

\part{Ontwerp}
\label{part:ontwerp}

%
% Systeemmodel
%

\chapter{Systeemmodel}
\label{chap:systeemmodel}

De eerste stap van het ontwerp was de identificatie van de verschillende deelsystemen, en op welke toestellen die te vinden zijn. Hiertoe hebben we eerst gekeken naar de verschillende taken die het systeem als een geheel moet vervullen. Zo zijn er natuurlijk de kiosken, die instaan voor het weergeven van de presentaties, en het verwerken van gebruikersinput. Om het systeem flexibel te houden, zullen we de kiosken zo inrichten dat zowel de configuratie als de weer te geven presentaties zich niet op voorhand op de kiosk bevinden, maar van een centrale server gehaald worden. Diezelfde centrale server kan dan ook voorzien in een beheersinterface, waarbij de status van de verschillende kiosken gevisualiseerd wordt, en de administrator eventueel bepaalde acties kan ondernemen. Al deze functionaliteit zullen we bundelen binnen het specifiek hiervoor ontworpen applicatie-raamwerk, waarvoor we ook een communicatieprotocol zullen voor moeten definiëren

Het is voor de administrator echter ook belangrijk om zicht te hebben op de status van de overige componenten van de kiosk, zoals het besturingssysteem, of de hardware. Omdat er voor dit vrij generiek probleem al verschillende bestaande oplossingen zijn, zullen we dit beschouwen als een apart deelsysteem, los van de hierboven omschreven applicatieraamwerk.

\begin{figure}
	\includegraphics[width=\textwidth]{diagrammen/ontplooiingsdiagram}
	\caption{Ontplooiingsdiagram}
\end{figure}

In de volgende hoofdstukken zullen we nu elk van deze deelsystemen, en al wat daar bij hoort, tot in details uitwerken.

%
% Applicatie
%

\chapter{Applicatie}
\label{chap:applicatie}

\section{Taken}

\section{Voorstellingen}

\section{Applicatieprotocol}

\subsection{Service discovery}

\subsection{Data uitwisseling}

\subsection{Remote procedure}

\section{Server}

\section{Kiosk}


%
% Monitoring
%

\chapter{Monitoring}
\label{chap:monitoring}


%
% Hardware
%

\chapter{Hardware}
\label{chap:hardware}

\input{4_server.tex}
\part{Kiosk}
\label{part:kiosk}
\part{Invoering}
\label{part:invoering}

\end{empfile}
\end{document}
